\documentclass[tikz]{standalone}
\usepackage{tikz-3dplot}
\usetikzlibrary{angles,arrows,backgrounds}
\tikzstyle{background grid}=[draw, step=0.2cm, gray, very thin]

% Colors
\colorlet{veccol}{black!30}     % unrotated vectors
% Reference frame axis styles
\tikzstyle{vec}=[-stealth,veccol,line cap=round]
\tikzstyle{rightangle}=[draw=veccol,very thin,angle radius=4mm,
pic text=.,pic text options={veccol},anchor=west]
\tikzstyle{dimline}=[Bar-Bar,dash pattern=on 2pt off 1pt,very thin,
veccol,line cap=round]

\begin{document}

\tdplotsetmaincoords{70}{135}
\begin{tikzpicture}[tdplot_main_coords, scale=1.5,
  every node/.style={scale=0.4}] % show background grid]
  % Macros for key quantities
  \pgfmathsetmacro{\psiangle}{50}
  \pgfmathsetmacro{\qlen}{0.5}
  \pgfmathsetmacro{\px}{1}     % V x component
  \pgfmathsetmacro{\py}{0}     % V y component
  \pgfmathsetmacro{\pz}{0}     % V z component
  % Coordinates
  \coordinate (O) at (0,0,0);
  \coordinate (Q) at (0,0,\qlen);
  \coordinate (V) at (\px,\py,\pz);

  % Create rotated frame and rotated vector
  \tdplotsetrotatedcoords{\psiangle}{0}{0}
  % Get coordinates of rotated V in main coordinate system
  \tdplottransformrotmain{\px}{\py}{\pz}
  \coordinate (Vr) at (\tdplotresx,\tdplotresy,\tdplotresz);
  % Main frame
  \draw[vec] (O) -- (Q) node[anchor=east]{\(q\)};
  \draw[vec] (O) -- (V) node[anchor=south]{\(v\)};
  \draw[vec,black] (O) -- (Vr) node[anchor=south west]{\(\hat{v}\)};
  \coordinate (ctr) at (barycentric cs:O=1,V=0.3,Vr=0.4);
  % Right angle indicator
  \path (Q) -- (O) -- (ctr)
  pic [rightangle] {right angle=Q--O--ctr};
  % Angle arc between vector and rotated vector
  \tdplotdrawarc[->,veccol,very thin]{(O)}{0.5}{0}{\psiangle}
  {anchor=south west}{\(\alpha\)};
  % q dimension line
  \draw[dimline] (O) ++(1.5mm,0) --
  node [anchor=west] {\(\sin(\alpha/2)\)} +(0,0,\qlen);
\end{tikzpicture}

\end{document}


%%% Local Variables:
%%% mode: latex
%%% TeX-master: t
%%% End:
