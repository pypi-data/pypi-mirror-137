\section{Entscheidungen unter Unwissenheit II}

In dieser Woche werden wir den Begriff des kardinalen Nutzen (bzw. des
"`Neumann-Morgensternschen"' Nutzens) einführen und einige weitere
Entscheidungsregeln kennen lernen, die auf diesem Nutzenkonzept beruhen. Aus
didaktischen Gründen wird erst ein Beispiel besprochen, in dem bereits der
kardinale Nutzen\footnote{Genaugenommen handelt es sich bei dem folgenden
Beispiel um einen kardinalen Wert, nämlich der Geldwert} vorausgesetzt wird und
erst danach der kardinale Nutzenbegriff selbst eingeführt.

\subsection{Die Minimax-Bedauerns-Regel}
\label{MinimaxRegret}

Von den bisher besprochenen Entscheidungsregeln ist die Maximin-Regel
wahrscheinlich die einleuchtendste und sinnvollste, aber wir haben auch schon
ein Beispiel kennen gelernt, bei dem ihre Anwendung möglicherweise nicht
sinnvoll wäre, und man kann weitere Beispiele konstruieren, bei denen das noch
deutlicher der Fall ist, z.B. das folgende:

\begin{center}
\label{MinimaxBedauernsRegel}
\begin{tabular}{c|c|c|}
\multicolumn{1}{c}{} & \multicolumn{1}{c}{$S_1$} & \multicolumn{1}{c}{$S_2$} \\
\cline{2-3}
$A_1$ & € 1,25 & € 1,50 \\ \cline{2-3}
$A_2$ & € 1,00 & € 50.000 \\ \cline{2-3}
\end{tabular}
\end{center}

Nach der Maximin-Regel müsste die Entscheidung zugunsten der Handlung $A_1$
ausfallen. Aber ist es sinnvoll, sich die Chance auf € 50.000 entgehen zu lassen,
nur um einen möglichen Verlust von 25 Cent zu vermeiden? Wenn man nicht gerade
eine Geschichte erfindet, bei der von diesen 25 Cent Leben und Tod abhängen,
erscheint das mehr als zweifelhaft.\marginline{Minimax-Bedauerns-Regel}  Um
Situationen wie dieser gerecht zu werden, gibt es eine Regel, die darauf zielt,
"`verpasste Chancen"' zu vermeiden. Diese Regel ist die {\em
Minimax-Bedauerns-Regel} (wohlbemerkt: diesmal heißt es "`Minimax"' nicht
"`Maximin"'!). Bei dieser Regel leitet man von der ursprünglichen Tabelle
zunächst eine Bedauernstabelle ab, die für jede Entscheidung und jedes
möglicherweise eintretende Ereignis (bzw. jeden möglichen Weltzustand) die Größe
der verpassten Chance beziffert. Dann wählt man diejenige Entscheidung aus, bei
der die größtmögliche verpasste Chance am kleinsten ist. Die Einträge in der
Bedauernstabelle erhält man, indem man jeden Wert in der Tabelle vom Maximalwert
derselben Spalte abzieht. Für das Beispiel von eben würde die Bedauernstabelle
dann so aussehen:

\begin{center}
\begin{tabular}{c|c|c|}
\multicolumn{1}{c}{} & \multicolumn{1}{c}{$S_1$} & \multicolumn{1}{c}{$S_2$} \\
\cline{2-3}
$A_1$ & € 0    & € 49.998,50 \\ \cline{2-3}
$A_2$ & € 0,25 & € 0 \\ \cline{2-3}
\end{tabular}
\end{center}

Das maximale Bedauern für die Handlung $A_1$ würde also mit € 49.998,50 zu
beziffern sein, während bei der Wahl von $A_2$ schlimmstenfalls ein Verlust
von 25 Cent verschmerzt werden müsste. Um das maximale Bedauern zu minimieren,
muss nach der Minimax-Bedauernsregel also die Handlung $A_2$ gewählt werden.

Ähnlich wie die die Maximin-Regel kann man die Minimax-Be\-dauerns\-regel auch
{\em lexikalisch} mehr\-fach hintereinander anwenden, wenn nicht gleich bei der
ersten Anwendung eine eindeutige Entscheidung getroffen werden kann.

An dieser Stelle könnte jedoch ein Einwand erhoben werden: Beim Übergang von der
Entscheidungstabelle zur Bedauernstabelle haben wir bestimmte Einträge in der
Tabelle voneinander subtrahiert. \marginline{Rechnen mit Nutzenwerten?}
Da es sich um Geldbeträge handelte, war das
denkbar unproblematisch, denn jeder wird zugeben, dass man mit Geldbeträgen
rechnen kann, und dass man sinnvollerweise davon sprechen kann dass € 3 dreimal
so viel Wert sind wie € 1. Aber was ist, wenn wir es nicht mit Geldbeträgen,
sondern wie zuvor mit ordinalen Nutzenwerten zu tun? Den vergleichsweise
voraussetzungsarmen Begriff des ordinalen Nutzens haben wir ja gerade deshalb
eingeführt, weil man mit anderen Werten als Geldbeträgen nicht unbedingt
Rechnungen durchführen kann, selbst wenn sich die Größe des Wertes noch
unterscheiden lässt. (Beispiel: Die meisten Menschen würden wohl zustimmen, dass
Bier und Würstchen leckerer sind als Brot und Wasser, aber es wäre Unsinn zu
sagen, sie sind genau dreimal so lecker.) Wenn wir eine Bedauernstabelle mit
ordinalen Nutzenwerten berechnen würden, dann würde sich das Ergebnis, das bei
der Anwendung der Minimax-Bedauerns-Regel herauskäme ändern, wenn wir die
Nutzenwerte durch ordinal transformierte Nutzenwerte ersetzen, was bei einer
robusten Entscheidungsregel nicht vorkommen sollte. Daher müssen wir entweder auf
die Anwendung der Minimax-Bedauerns-Regel verzichten, oder wir dürfen sie nur
dort anwenden, wo wir einen stärkeren Nutzenbegriff vorausetzen dürfen, wie er
z.B. implizit den in den vorhergehenden Beispielen verwendeten Geldwerten zu
Grunde liegt. Der schwächstmögliche stärkere Nutzenbegriff (stärker im Vergleich
zum ordinalen Nutzen), der es uns erlaubt die Minimax-Bedauerns-Regel anzuwenden,
ist der Begriff des kardinalen Nutzens. 

Bevor wir jedoch auf den Begriff des kardinalen Nutzens eingehen, soll aber
noch auf eine besondere Eigenschaft der Minimax-Bedauerns-Regel hingewiesen
werden, die unter Umständen auch als ein Einwand gegen diese Regel begriffen
werden kann: Die Minimax-Bedauerns-Regel verletzt nämlich -- ebenso wie übrigens auf die
Rangordnungsregel aus Kapitel \ref{Rangordnungsregel} -- das Prinzip der {\em
paarweisen Unabhängigkeit} oder auch "`Unabhängigkeit von dritten
Alternativen"'.\footnote{Die dafür häufig auch verwendete Bezeichnung
"`Unabhängigkeit von irrelevanten Alternativen"' ist wegen ihrer Suggestivität
irreführend. Es ist nämlich keineswegs immer so, dass dritte Alternativen
grundsätzlich irrelevant sind.}
\marginline{Un\-ab\-häng\-ig\-keit von dritten Alternativen}
Fügt man den bestehenden Handlungsalternativen eine Handlungsalternative hinzu,
so kann das selbst dann zu einer Änderung der Entscheidung führen, wenn die neu
hinzugefügte Alternative nach der Minimax-Bedauerns-Regel sowieso nicht gewählt
werden würde. Beispiel:

\begin{center}

\setlength{\parskip}{0.5cm}

\begin{tabular}{c|p{1cm}|p{1cm}|p{1cm}|cc|p{1cm}|p{1cm}|p{1cm}|}
\multicolumn{1}{c}{} & \multicolumn{3}{c}{Entscheidungstabelle} &
\multicolumn{2}{c}{} & \multicolumn{3}{c}{"`Bedauerns"'-tabelle}
\\ \cline{2-4} \cline{7-9}
$A_1$ & 0 & 10 & 4    & & $A_1$ & 5 & 0 & 6  \\ 
\cline{2-4} \cline{7-9} 
$A_2$ & 5 &  2 & 10   & & $A_2$ & 0 & 8 & 0  \\ 
\cline{2-4} \cline{7-9}
\end{tabular}

\begin{tabular}{c|p{1cm}|p{1cm}|p{1cm}|cc|p{1cm}|p{1cm}|p{1cm}|}
\cline{2-4} \cline{7-9}
$A_1$ & 0 & 10 & 4  & & $A_1$ & 10 & 0 & 6  \\ 
\cline{2-4} \cline{7-9} 
$A_2$ & 5 &  2 & 10 & & $A_2$ & 5 &  8 & 0  \\ 
\cline{2-4} \cline{7-9}
$A_3$ & 10 &  5 & 1 & & $A_3$ & 0 &  5 & 9 \\ 
\cline{2-4} \cline{7-9}
\end{tabular}

{\tiny Quelle: Michael D. Resnik: Choices. An Introduction to Decision Theory,
Minnesota 2000, S. 31.}
\end{center}

Die Alternative A3 hat nach der Minimax-Be\-dau\-erns-Re\-gel keine Chance
ge\-wählt zu werden. Dennoch übt ihre Präsenz Einfluss darauf aus, welche der beiden
anderen Handlungsalternativen nach der Minimax-Bedauerns-Regel gewählt wird. Ist
die Alternative A3 abwesend, so ist die Handlung A1 nach der
Minimax-Bedauerns-Regel die beste Handlung. Fügt man die Alternative A3 hinzu, so
ist A2 die beste Handlung.

Sollte man die Abhängigkeit von dritten Alternativen als eine Schwäche der
Minimax-Bedauerns-Regel ansehen? Das hängt wiederum sehr davon ab, in welchem
Zusammenhang die Entscheidungsregel angewandt wird. Da das Prinzip besonders in
der Sozialwahltheorie eine große Rolle spielt, dazu einige Beispiele:

\begin{enumerate}
\label{dritteAlternativen}
  \item Resnik erzählt dazu in etwa die folgende Geschichte \cite[S. 40]{resnik:1987}:
\marginline{Beispiele für die mögliche Relevanz von dritten Alternativen}
Stellen Sie sich vor, Sie sitzen in einem Restaurant und überlegen, ob Sie lieber
ein Steak oder ein vegetarisches Gericht bestellen wollen. Eigentlich mögen Sie
lieber Steak, aber da das Restaurant einen etwas heruntergekommenen Eindruck
macht, haben Sie wegen der Fleischzubereitung so ihre Bedenken und tendieren eher
zu einer vegetarischen Speise. Nun erzählt Ihnen die Dame vom Nebentisch, dass
sie gerade ein vorzügliches Schnitzel gegessen hat. Sie selbst -- nehmen wir an
-- mögen zwar kein Schnitzel, aber obwohl diese Alternative für Sie
"`irrelevant"' ist, wissen Sie nun, dass Sie der Fleischzubereitung in diesem
Restaurant vertrauen können, und bestellen doch das Steak.

  \item Frau Schmidt möchte ein Auto kaufen. Sie legt Wert darauf, dass es das
  teuerste und, wenn nicht das teuerste, dann doch wenigstens dass schnellste
  Auto von der ganzen Stadt ist. Also entscheidet sie sich gegen einen Porsche
  und für einen S-Klasse Mercedes, weil der teurer ist. Jetzt efährt sie aber,
  dass ihre Nachbarin Frau Klein sich kürzlich einen Rolls-Royce zugelegt hat. 
  Einen Rolls-Royce kann sich Frau Schmitt aber sowieso nicht leisten. Da sie
  dann aber statt des teuersten wenigstens das schnellste Auto haben will,
  kauft sie sich nun doch nicht den Mercedes, sondern lieber den Porsche.

  Ihre Wahl zwischen Porsche und Mercedes ist also nicht unabhängig von dritten
  Alternativen, auch diese für Frau Schmitt sowieso nicht in Frage kommen, wie in
  diesem Fall der Rolls Royce.
  
  \item {\em Machinas Paradox}:
\label{machinasParadox}
Angenommen, eine Person habe die Wahl zwischen zwei Lotterien:
\begin{enumerate}
  \item Lotterie: 99\% Chance eine Reise nach Venedig zu gewinnen, 1\%
  Chance eine Filmvorführung über Venedig zu gewinnen.
  \item Lotterie: 99\% Chance eine Reise nach Venedig zu gewinnen, 1\%
  Chance zu Hause zu bleiben.
\end{enumerate}
Im Sinne der Theorie müsste die erste Lotterie eindeutig bevorzugt werden, wenn
man annimmt, dass einen Film über Venedig anzuschauen allemal interessanter
ist, als zu Hause zu sitzen. Andererseits ist es durchaus plausibel sich
vorzustellen, dass angesichts der sehr großen Chance eine Reise nach Venedig zu
gewinnen, es doch noch erträglicher ist zu Hause zu bleiben, wenn man
die Chance verpasst, als sich dann auch noch einen herrlichen Film über Venedig
anschauen zu müssen. 

Wenn man diese Argumentation akzeptiert, dann zeigt das Beispiel einmal mehr,
dass die Annahme der Unabhängigkeit von dritten Alternativen, z.B. auf Grund
solcher psychologischen Faktoren wie des Bedauerns, nicht immer zwingend oder
auch nur glaubwürdig ist. Oder gäbe es vielleicht eine Möglichkeit, das Beispiel
durch eine entsprechende Problemspezifikation, z.B. durch Einbeziehen des Bedauernsfaktors
in die Konsequenz der Entscheidung, doch noch mit der Theorie zu vereinbaren?
(Aufgabe \ref{AufgabeDritteAlternative}!)

\item {\em Mögliche Bedeutung der Rangordnung} \cite[S. 81]{mackie:2003}: Wie
bereits erwähnt ist auch die Rangordnungsregel nicht mit dem Prinzip der
paarweisen Unabhängigkeit vereinbar. Hält man Entscheidungsregeln wie die
Minimax-Bedauernsregel oder die Rangordnugnsregel für sinnvoll, so kann man diese
Unvereinbarkeit statt gegen bestimmte Entscheidungsregeln umgekehrt auch gegen
das Prinzip der paarweisen Unabhängigkeit ausspielen. 

Vorgreifend auf die Sozialwahltheorie sei zur Illustration der möglichen Relevanz
der Rangordnung von Präferenzen und damit auch der Relevanz von dritten
Alternativen folgendes Beispiel diskutiert: Angenommen Napoleon habe die
Präferenzen $b \succ a \succ c \succ d \succ e$ und Josephine $a \succ b \succ c
\succ d \succ e$. Es sei weiterhin angenommen, dass Napoleon und Josephine sich
darauf einigen müssten, ob sie gemeinsam $a$ oder $b$ wählen wollen, und dass
Napoleon sich nach langwierigen Diskussionen schließlich durchgestezt habe sie
gemeinasm $b$ wählen.

Nun erhält Josephine eine Nachricht, die dazu führt, dass sie ihre Präferenzen
dergestalt abändert, dass die Alternative $b$ nun für sie an die letzte Stelle
rückt, so dass sie nun die Präferenzen $a \succ c \succ d \succ e \succ b$ hat. 

Josephine teilt dies Napoleon mit, und bittet darum, auf Grund der geänderten
Umstände die gemeinsame Entscheidung noch einmal zu überdenken. Napoleon
antwortet ihr jedoch mit dem Hinweis auf das Prinzip der Unabhängikeit von
"`irrelevanten"' Alternativen, dass dies nicht erforderlich sei, da sich
Josephines Präferenzen bezüglich $a$ und $b$ duch die neu eingetretenen Umstände nicht
geändert hätten, so dass sie die Entscheidung zwischen $a$ und $b$ gar nicht
beeinflussen dürften.

Sofern man Napoleons Antwort als unverschämt empfindet, ist dieses Beispiel ein
Gegenbeispiel gegen die generelle Gültigkeit des "`Prinzips des Unabhängigkeit
von dritten Alternativen"'. Das Beispiel zeigt, dass das Prinzip der
Unabhängkeit von dritten Alternativen uns zwingt, von der Information über die
Rangordnung der beiden zur Entscheidung anstehenden Alternativen innerhalb einer
größeren Menge von Alternativen zu abstrahieren. Aber unter Umständen könnte
diese Information wichtig sein, z.B. indem sie die Intensität einer Präferenz
ausdrückt und sofern man der Ansicht ist, dass die Intensität der
individuellen Präferenzen bei der Diskussion über gemeinsame Entscheidungen wie
der von Napoleon und Josephine mitberücksichtigt werden sollte.

Man kann es auch so formulieren: Eine dogmatische Festlegung auf das Prinzip
der Unabhängikeit von dritten Alternativen würde Entscheidungsprobleme wie das
von Napoleon und Josephine aus dem Anwendungsbereich der Entscheidungstheorie
ausschließen.


\end{enumerate}
 
Wie man sieht können dritte Alternativen sehr wohl relevant für die relative
Bewertung der anderen Alternativen sein. Insofern muss die Abhängigkeit von
dritten ("`irrelevanten"') Alternativen nicht unbedingt als eine Schwäche der
Entscheidungsregel aufgefasst werden. Aber es gibt andere Situationen, wo das
durchaus der Fall sein kann, etwa bei Wahlen oder Abstimmungen, deren Ergebnis
unter Umständen dadurch manipuliert werden könnte, dass man weitere, scheinbar
irrelevante Alternativen zur Abstimmung stellt.\footnote{Theoretische Beispiele
findet man in der entsprechenden Fachliteratur unter den Stichworten "`Paradox of
Voting"' und "`Agenda Setting"' in Fülle \cite[S. 112ff.]{mueller:2003}. Die
empirische Relevanz des vermeintlichen Problems zyklischer Mehrheiten wird jedoch
inzwischen sehr stark in Zweifel gezogen \cite[S. 147ff.]{green-shapiro:1994}.
Praktisch spielen die Formen der Abstimmungsmanipulation, die in der Public
Choice Literatur so ausführlich erörtert werden, keine Rolle, während andere, die
womöglich viel wichtiger sind, von den Autoren der Public Choice Literatur nicht
beachtet werden.} Insgesamt kann man sagen, dass das Prinzip der
Unabhängigkeit von dritten Alternativen bzw. der paarweisen Unabhängigkeit nur
dann aufgestellt werden sollte, wenn man zuvor sichergestellt hat, dass für die
Entscheidung zwischen jedem Paar von Alternativen (bzw. für die relative
Bewertung von jedem Paar von Alternativen) die Verfügbarkeit der anderen
Alternativen tatsächlich irrelevant ist. In einer Entscheidungssituation, wo
dies nicht der Fall ist, kann eine Theorie, die dieses Prinzip als Axiom
einführt, nicht ohne Einschränkungen angewendet werden.


\subsection{Kardinaler Nutzen}
\label{KardinalerNutzen}
Der Grundgedanke der "`Minimax-Bedauerns-Regel"' besteht darin, eine Entscheidung
zu finden, bei der der maximal mögliche Verlust (je nach eintretenden
Zufallsereignissen) minimiert wird. Da wir diese Regel auf ein Beispiel mit
Geldwerten angewendet haben, konnten wir die Verluste relativ bedenkenlos als die
Differenz zwischen entgangenem Gewinn und erhaltenem Gewinn bestimmen. Aber wie
sollen wir eine solche Regel wie die "`Minimax-Bedauerns-Regel"' anwenden, wenn
die (möglichen) Ergebnisse eines Entscheidungsproblems keine Geldwerte sind? Die
Ihnen zugeordneten Nutzenwerte spiegeln dann -- nach dem Konzept des {\em
ordinalen Nutzens} -- nur eine Rangordnung zwischen den möglichen Ergebnissen des
Entscheidungsprozesses entsprechend den Präferenzen wieder.\marginline{Grenzen
des ordinalen Nutzens} Das Ergebnis der Anwendung einer Entscheidungsregel sollte
also auch nur von der Rangordnung der Nutzenwerte nicht aber von den -- solange
die Ordnung erhalten bleibt -- willkürlich wählbaren Zahlenwerten abhängen, die
diese Ordnung auf einer Nutzenskala wiedergeben. Betrachten wir als Beispiel
einmal folgende beiden Nutzenskalen, die den Ergebnissen $x, y, z$ jeweils einen
bestimmten Nutzen zuordnen. ({\em x}, {\em y} und {\em z} sollen dabei
irgendwelche möglichen Resultate irgendeines Entscheidungsprozesses sein, z.B.
könnten sie für die Resultate {\em frustriert}, {\em gelangweilt}, {\em erfreut}
aus dem Beispiel auf Seite \pageref{AngelnBeispiel} stehen.)

\begin{center}
\begin{tabular}{cc|c|cccc|c|c}
& x  &  y  &  z  &  &  &  x  &  y  &  z  \\ \cline{2-4} \cline{7-9}
\raisebox{1.5ex}[-1.5ex]{Nutzenskala {\bf u()}} 
& 1  &  2  &  3  &  &  
\raisebox{1.5ex}[-1.5ex]{Nutzenskala {\bf v()}}
&  1  &  4  &  9 
\end{tabular}
\end{center}

Beide Skalen geben offenbar denselben ordinalen Nutzen wieder, da $u(z) > u(y) >
u(x)$ und ebenso $v(z) > v(y) > v(x)$. Betrachtet man allerdings die Differenzen, so
fällt auf, dass $u(z) - u(y) = u(y) - u(x)$, während $v(z) - v(y) > v(y) -
v(x)$. Würden diese Nutzenwerte bei einem Entscheidungsproblem auftauchen, so
könnte es geschehen, dass man bei Anwendung der Minimax-Bedauernsregel je
nachdem, ob man die Nutzenfunktion {\em u} oder die Nutzenfunktion {\em v} zur
Darstellung der Präferenzen heranzieht, zu einer anderen
Entscheidungsempfehlung kommt. Genau das dürfte aber nicht geschehen, da {\em
u} und {\em v} nur unterschiedliche Darstellungen desselben {\em ordinalen}
Nutzens sind. Welche Auswege könnte man sich aus dieser misslichen Situation
denken:

\begin{enumerate}
  \item Angesichts des Beispiels (Seite \pageref{MinimaxBedauernsRegel}), mit
  dem wir die Mini\-max-\-Bedauernsregel eingeführt haben, könnte man auf die
  naheliegende Idee verfallen, dass man diese Regel nur in solchen Fällen
  anwenden kann, in denen die Ergebnisse des Entscheidungsprozesses monetäre
  Auszahlungen sind. Das hätte allerdings zwei Nachteile: 1) Die Anwendbarkeit
  der Regel würde dabei auf eine vergleichsweise kleine Menge von
  Entscheidungsproblemen eingeschränkt. 2) In vielen Situationen, in denen in
  irgendeiner Form monetäre Auszahlungen vorkommen, geben die monetären
  Auszahlungen nicht unmittelbar den damit assoziierten Nutzen wieder.
  Hanldungsleitend und damit entscheidungsrelevant ist jedoch
  der Nutzen und nicht der Geldwert.\marginline{Unterschied von Nutzen und
  Geldwert} Ein Beispiel daür, dass Nutzen und Geldwert sich nicht decken müssen
  ist das folgende: 2.000 Euro sind doppelt so viel Geld wie 1.000 Euro. Aber der
  zusätzliche Nutzen, den man von  2.000 Euro Monatsgehalt 
  gegenüber 1.000 Euro Monatsgehalt gewinnt, ist sicherlich geringer 
  als der zusätzliche Nutzen 
  von 1.000 Euro gegenüber 0 Euro Gehalt.
  
  \item Eine andere denkbare Alternative wäre die Aufstellung einer {\em
  qualitativen Bedauernstabelle}. Dazu müsste man zunächst einmal die
  {\em Differenzergebnisse} bestimmen, worunter man zusammengesetzte Ergebnisse
  aus einem nicht eingetretenen und einem statt dessen eingetretenen
  Ergebnis verstehen kann. (In dem Beispiel des Küstenbesuchers aus der ersten
  Vorlesung (Seite \pageref{AngelnBeispiel}), in dem die möglichen Resultate 
  {\em frustriert}, {\em gelangweilt}, {\em erfreut}
  waren, würden sich daraus die Differenzereignisse {\em frustriert statt
  bloß gelangweilt}, {\em gelangweilt statt erfreut} und {\em frustriert statt
  erfreut} ergeben.) Weiterhin müsste man ein neutrales Differenzereignis
  definieren, welches die Stelle der 0 in der aus Nutzenwerten gewonnen
  Bedauernstabelle einnimmt. Dieses neutrale Differenzergebnis könnte man z.B.
  als "`Unter gegebenen Umständen so gut wie möglich"' bezeichnen oder ähnlich.
  Schließlich müsste man die Präferenzen bezüglich der Differenzergebnisse
  bestimmten, denen man dann eine neue ordinale Nutzenfunktion zuweisen könnte.
  Die Bestimmung des minimalen größten Bedauerns erfolgt wie zuvor beschrieben
  (Siehe Abschnitt \ref{MinimaxRegret}). Der Nachteil dieses Vorgehens besteht
  erstens darin, dass die Präferenzordnung für eine weitere Ergebnismenge,
  nämlich die Menge der Differenzergebnisse, bestimmt werden muss, und zweitens
  darin, dass sich dieses Verfahren tatsächlich nur auf die
  Minimax-Bedauerns-Regel anwenden lässt, nicht mehr aber auf
  die meisten weiteren Entscheidungsregeln,
  die wir gleich noch kennen lernen werden. In den Fällen aber, in denen wir
  nicht die gleich zu besprechende Neuman-Morgensternsche Nutzenfunktion bilden
  können (d.h. in den Fällen, in denen wir aus empirisch-sachlichen
  Gründen höchstens einen {\em ordinalen} Nutzen voraussetzen dürfen) bleibt die
  Bildung einer qualitativen Bedauernstabelle die einzige Alternative.
    
  \item Schließlich kann man versuchen, ein "`stärkeres"' Nutzenkonzept als das
  des ordinalen Nutzens zu Grunde zu legen.\marginline{kardinaler Nutzen}
  Bei einem solchen Nutzenkonzept
  müsste nicht nur die Ordnung der Nutzenwerte unter einer Transformation
  erhalten bleiben sondern mindestens auch die Ordnung beliebiger Differenzen
  von Nutzenwerten. Stärker ist ein solches Nutzenkonzept in dem Sinne,
  dass die Nutzenwerte dann mehr Informationen enthalten als nur die
  Information über die Ordnung der Präferenzen. Das bedeutet aber auch, dass
  ein solches Nutzenkonzept empirisch schwerer zu rechtfertigen ist, und dass
  der empirische Anwendungsbereich eines solches Nutzenkonzepts
  kleiner sein wird als der des ordinalen Nutzens. Um
  die Ordnung der Differenzen zu erhalten, ist es aber andererseits noch längst
  nicht erforderlich, den konkreten Zahlenwerten der Nutzenfunktion eine
  eindeutige Interpretation zu geben, wie dies bei der Zuweisung von Geldwerten
  der Fall wäre. Gesucht ist also ein möglichst schwaches (und damit 
  empirisch immer noch möglichst breit anwendbares) Nutzenkonzept, das
  aber stärker ist als das des Ordinalen Nutzens. 
  Ein solches Nutzenkonzept ist das des {\em
  kardinalen} bzw. des {\em Neumann-Morgensternschen Nutzens}.
\end{enumerate}

Das, was wir eben eher intuitiv die "`Stärke"' eines Nutzenkonzepts genannt
haben, ist dadurch bestimmt, unter welcher Art von Transformationen man zwei
Nutzenfunktionen als {\em äquivalent}, d.h. denselben Nutzen ausdrückend,
betrachtet. (Man kann es also nicht den Nutzenfunktionen also solchen ansehen, ob
sie einen ordinalen oder kardinalen Nutzen ausdrücken. Sondern erst durch den
Vergleich von Nutzenfunktionen und der Festlegung der Bedingungen ihrer
Äquivalenz oder Nicht-Äquivalenz wird dies bestimmt.) Beim {\em ordinalen Nutzen}
wurden alle Nutzenfunktionen als äquivalent betrachtet, die durch
"`ordnungserhaltende"' Transformationen ineinander überführt werden können.
"`Ordnungserhaltend"' sind alle streng monoton steigenden Abbildungen. Der {\em
kardinale Nutzen} ist nun dadurch definiert, dass zwei Nutzenfunktionen als
äquivalent betrachtet werden, wenn man sie durch {\em positive lineare
Transformationen} ineinander überführen kann. Positive lineare Transformationen
sind alle Transformationen der Form:

\begin{displaymath}
u(x) = ax + b, \qquad a > 0
\end{displaymath}

Man betrachte unter diesem Gesichtspunkt einmal die folgenden, in
Tabellen dargestellten Nutzenfunktionen:

\begin{center}
\begin{tabular}{cc|c|cccc|c|cccc|c|c|}
& x  &  y  &  z  &  &  &  x  &  y  &  z & & & x & y & z \\ 
\cline{2-4} \cline{7-9} \cline{12-14}
\raisebox{1.5ex}[-1.5ex]{{\bf u()}} 
&  1  &  2  &  3  &  &  
\raisebox{1.5ex}[-1.5ex]{{\bf v()}}
&  1  &  4  &  9 &  &
\raisebox{1.5ex}[-1.5ex]{{\bf w()}}
&  1  &  3  &  5
\end{tabular}
\end{center}

Alle drei Nutzenfunktionen geben denselben ordinalen Nutzen wieder, aber nur die
Funktionen {\em u} und {\em w} geben denselben kardinalen Nutzen wieder, da $w(x)
= 2u(x)-1$. Weiterhin kann man sich leicht überlegen, dass zwei Nutzenfunktionen,
die denselben kardinalen Nutzen darstellen, immer auch denselben ordinalen Nutzen
repräsentieren, denn positive lineare Transformationen sind immer auch
ordnungserhaltende Transformationen. Umgekehrt gilt dasselbe aber nicht, wie die
Tabelle oben zeigt. Kardinale Nutzenskalen sind "`feinkörniger"' als ordinale
Nutzenskalen.
\marginline{Erhalt der Ordnung von Nutzendifferenzen}
Und sie erhalten, wie erwünscht nicht nur die Ordnung der Nutzenwerte sondern
auch die Ordnung der Differenzen von Nutzenwerten, denn seien $x,y,z,w \in
\mathbb{R}$ beliebige Nutzenwerte und sei $u(x) = ax + b$ mit $a,b \in
\mathbb{R}, a > 0$ eine positive lineare Transformation, dann:
\begin{eqnarray*}
  x - y & > & z - w  \\
  a(x-y) & > & a(z-w)  \\
  a(x-y) + b - b & > & a(z-w) + b - b \\
  (ax + b) - (ay + b) & > & (az + b) - (aw + b) \\
  u(x) - u(y) & > & u(z) - u(w) 
\end{eqnarray*}
Dasselbe gilt, wenn man statt des Ungleichheitszeichens ein Gleichheitszeichen
einsetzt, womit der Erhalt der Ordnung von Nutzendifferenzen unter positiv
linearer Transformation bewiesen ist. Positive lineare Transformationen haben 
darüber hinaus die Eigenschaft, dass sie nicht bloß die
Ordnung der Differenzen von Nutzenwerten erhalten, sondern auch die Quotienten
der Differenzen:\marginline{Erhalt der Quotienten von Nutzendifferenzen}
\[ \frac{u(x) - u(y)}{u(z) - u(w)} = 
   \frac{(ax + b) - (ay + b)}{(az + b) - (aw + b)} = 
   \frac{a(x - y) + b-b}{a(z - w) + b-b} = 
   \frac{x - y}{z - w} \]
Diese Eigenschaft wird später noch für uns wichtig werden wird. Erfüllt eine
Skala, wie in diesem Fall die kardinale Nutzenskala, diese Eigenschaft, so nennt
man sie auch eine {\em Intervallskala}. Zur besseren Übersicht sollen im
folgenden kurz einige der wichtigsten Skalentypen aufgelistet werden, die in
der Wissenschaft von Bedeutung sind.

\subsubsection{Exkurs: Skalentypen}

Skalen dienen dazu abgestufte Größen darzustellen. Nun gibt es unterschiedliche
Grade, in denen irgendwelche Größen abgestuft sein können. (Mit dem ordinalen
und dem kardinalen Nutzen haben wir schon zwei unterschiedliche Abstufungsgrade
kennen gelernt.) Diese unterschiedlichen Abstufungsgrade spiegeln sich in
den verschiedenen Skalentypen wieder. Die Skalentypen sind dabei von gröberen zu
immer feineren Skalentypen geordnet. (Vgl. zum folgenden \cite[S.
73ff.]{schurz:2006}) 

Das gröbste bzw. "`niedrigste"' Skalenniveau, das man sich vorstellen kann, ist
das einer {\bf Nominalskala}\marginline{Nominalskala}. Bei einer Nominalskala
wird die gegebene Größe lediglich in eine von mehreren begrifflichen Kategorien 
eingordnet, ohne dass
zwischen diesen Kategorien eine Ordnung des Mehr- und Weniger besteht. Man
spricht deshalb auch von "`Kategorienskalen"' oder von
"`qualitativ-klassifikatorischen Begriffen"'. Ein Beispiel wäre etwa die
Zuordnung von Wirtschaftsunternehmen zu unterschiedlichen Wirtschaftssektoren wie
a) Landwirtschaft, b) Handel und Industrie, c) Dienstleistung. Die einzigen
Bedinungen, denen eine Nominalskala genügen muss, bestehen darin, dass die
Kategorien 1. {\em disjunkt} (kein Gegenstand kann unter mehr als eine Kategorie
gleichzeitig fallen) und 2. {\em exhaustativ} (jeder Gegenstand kann in
mindestens eine Kategorie eingeordnet werden) sein müssen. Eine wie auch immer
geartete Ordnungsbeziehung muss zwischen den Kategorien aber nicht bestehen. (Man
kann ja auch z.B. kaum sinnvollerweise sagen, dass Dienstleistung "`mehr"' oder
"`größer"' ist als Landwirtschaft. Allenfalls könnte man das von der Anzahl der
Beschäftigten oder dem erwirtschafteten Umsatz in dem entsprechenden Sektor
sagen.)

Das nächsthöhere Skalenniveau stellt die {\bf
Ordinalskala}\marginline{Ordinalskala/ Rangskala} (auch "`Rangskala"') dar. Im
Gegensatz zur Nominalskala werden hier die Merkmale bzw. die Objekte des
Gegenstandsbereichs in "`Ranggruppen"' eingeteilt, zwischen denen eine Höher- und
Niedriger-Beziehung besteht. (Für die präzise Definition einer solchen {\em
Quasi-Ordnungs}-Beziehung siehe Seite \pageref{Ordnungsaxiome}) Außer dem nun
schon bekannten ordinalen Nutzen, wäre ein weiteres Beispiel die Mohs-Skala aus
der Mineralogie, bei der die Härte von Mineralien danach geordnet wird, welches
Mineral welche anderen "`ritzt"' \cite[S. 75]{schurz:2006}.

Auf die Ordinalskala folgt in der Rangfolge die {\bf
Intervallskala}\marginline{Intervallskala}. Intervallskalen verfügen über eine
mehr oder weniger willkürlich gewählte Maßeinheit. Weder die Maßeinheit selbst
noch der Nullpunkt einer Intervallskala sind in irgendeiner Weise durch den
Gegenstandsbereich festgelegt. Voraussetzung ist jedoch, dass die auf einer
Intervallskala abgebildete Größe zahlenmäßig empirisch messbar ist. Die
Maßeinheit erlaubt es dann, Differenzen und Quotienten von Differenzen der
gemessenen Größe zu vergleichen. Beispiele sind denn auch Orts- und
Zeitmessungen, denn ob man das Jahr 0 auf Christi Geburt oder auf den Zeitpunkt
der Auswanderung Mohammeds nach Medina verlegt, ist eine Sache bloßer Konvention,
genauso wie es eine Konvention ist, dass der Nullmeridian in Greenwich liegt.
Trotzdem kann man Zeit- und Ortsdifferenzen sowie Quotienten von Differenzen
vergleichen (eine Stunde ist solange wie jede andere und drei Stunden sind
dreimal solange wie eine Stunde).

Die {\bf Verhältnisskala}\marginline{Verhältnisskala} schießlich unterscheidet
sich von der Intervallskala dadurch, dass nur noch die Maßeinheit willkürlich
festgelegt ist, der Nullpunkt aber durch die Wirklichkeit vorgegeben ist.
Beispiele dafür sind etwa Gewichtsskalen oder auch die Temperaturskala nach
Kelvin, die den Nullpunkt auf den "`absoluten Nullpunkt"' bei 273,15 Grad Celsius
verlegt. Auch Geldwerten liegt eine Verhältnisskala zu Grunde, denn der Nullpunkt
(d.h. wenn jemand gar kein Geld hat) ist ja in naheliegender Weise vorgegeben.

Schließlich kann man von allen vorhergehenden Skalen noch die {\bf
Absolutskala}\marginline{Absolutskala} unterscheiden, bei der man verlangen
müsste, dass auch die Maßeinheit selbst noch eine zwingende empirische
Interpretation hat. Dergleichen ist aber im Grunde nur bei einfachen Zählskalen
der Fall. Wenn man also z.B. von "`drei Äpfeln"' spricht, dann hat die Zahl drei
dabei einen ganz bestimmten empirischen Sinn und es ist nicht eine Frage der
Konvention ob man drei oder zwei sagt, wie es eine Frage der Konvention ist, ob
man eine Länge in Meter oder Fuß angibt.

Insgesamt ergibt sich also eine Abfolge von fünf Skalentypen: 

\begin{center}
{\small Nominalskala < Ordinalskala < Intervallskala < Verhältnisskala <
Absolutskala}
\end{center}

Im Anschluss an diese Auflistung von Skalentypen stellen sich zwei naheliegende
Fragen: Erstens: Sind das alle Skalentypen, die es gibt? Und zweitens: Wonach
richtet sich, welchen Skalentyp man verwenden kann oder soll?

Was die erste Frage betrifft, so sind die aufgeführten Skalentypen natürlich
längst nicht alle denkbaren Skalentypen. Einmal könnte man die Abfolge von
Skalentypen sehr wohl noch weiter verfeinern.\marginline{Mehr\-dim\-en\-sion\-ale
Skalen} Dann gibt es, was noch wichtiger ist, neben den hier aufgeführten
eindimensionalen Skalen auch mehrdimensionale Skalen. Zu diesen zählen
beispielsweise Farbskalen bzw. Farbräume. Im RGB-Farbraum etwa wird jede Farbe
durch ein 3-tupel des Rot-, Grün- und Blauwertes angegeben, aus denen die Farbe
nach dem Prinzip der additiven Mischung zusammengesetzt ist.

Was die zweite Frage betrifft,\marginline{Auswahl des Skalentyps} so richtet sich
die Verwendung eines bestimmten Skalentyps nach den {\em empirischen
Eigenschaften} der auf der Skala abgebildeten Größe und nach den {\em vorhandenen
Messmethoden}. So kann man die Länge deshalb auf einer Intervallskala messen,
weil wir mit dem "`Urmeter"' über einen entsprechenden Vergleichsmaßstab
verfügen. Bei der Härtemessung von Materialen nach der Mohs-Skala gibt es keinen
solchen Vergleichsmaßstab, so dass sie auch nicht auf einer Intervallskala,
sondern nur auf einer Ordinalskala angegeben werden kann.

Besonders schwierig gestaltet sich die Suche nach geeigneten Messmethoden und
damit die "`Metrisierung"'\marginline{Schwierigkeiten der Metrisierung in den
Sozialwissenschaften} (d.h. die Überführung von komparativen Begriffe in
quantitative mittels geeigneter Messmethoden) in den Sozialwissenschaften. Denn
während die verschiedenen Zahlenmengen von den natürlichen Zahlen bis hin zu den
komplexen Zahlen geradezu dafür geschaffen scheinen, die Zusammenhänge
auszudrücken, die die Naturwissenschaften untersuchen (dazu sehr eindrucksvoll
Penrose \cite[S. 51ff.]{penrose:2004}), weshalb man in diesem Bereich recht
eigentlich sagen darf, dass die Mathematik die Sprache der Natur ist, lassen sich
mathematische Gesetze für die Sozialwissenschaften vielfach nur unter erheblicher
Strapazierung der Begriffe einspannen. Diese Schwierigkeiten begegnen uns auch
beim Präferenzbegriff, denn während man die Annahme, dass es ordinale Präferenzen
(soll heißen: Präferenzen, die durch ordinale Nutzenfunktionen beschrieben
werden können) gibt, noch einigermaßen glaubwürdig rechtfertigen kann, und es zumindest
vorstellbar erscheint, die Ordnung von Präferenzen durch Befragung oder
Verhaltensbeobachtung halbwegs zuverlässig festzustellen, so ist dies bei der
Annahme kardinaler Präferenzen nur unter Schwierigkeiten möglich. Wenn man aber
annimmt, dass bei solchen Gegenständen, deren Wert sich durch Geld ausdrücken
lässt (also bei "`Waren"') der kardinale Nutzen einigermaßen mit dem Geldwert
korreliert, dann erscheint die Annahme nicht ganz abwegig, dass es so etwas wie
kardinale Präferenzen geben könnte.

Eine weitere Schwierigkeit, die mit der Beantwortung der Frage, welche Art von
Skala man zur Nutzenmessung verwenden darf, noch gar nicht berührt ist, ist die
ob Nutzenbewertungen immer nur jeweils für eine Person gültig sind, 
\marginline{Problem intersubjektiver Nutzenvergleiche} oder ob man
auch die Nutzenwerte unterschiedlicher Personen untereinander vergleichen darf
({\em intersubjektiver Nutzen}). In Bezug
auf solche Güter, deren Wert von den meisten Menschen gleich hoch geachtet wird
(z.B. Gesundheit, Leben, Wohlstand, Jugend etc.) erscheint ein intersubjektiver
Nutzenvergleich nicht abwegig. Ebenso erscheint ein intersubjektiver
Nutzenvergleich bei Gütern möglich, für die soziale Institutionen existieren, die
solche Nutzenvergleiche hervorbringen, wie das z.B. Märkte für Waren tun. Bei
anderen Gütern mag das nicht immer möglich sein. Mit den beiden Unterscheidungen
kardinaler Nutzen - ordinaler Nutzen und subjektiver Nutzen - intersubjektiver
Nutzen ergeben sich insgesamt vier Arten von Nutzenkonzepten:
\begin{center}
\begin{tabular}{cc|p{3.2cm}|p{3.2cm}|}
& \multicolumn{1}{c}{} & \multicolumn{2}{c}{Skalentyp} \\
&                  & \multicolumn{1}{c|}{\em ordinal}
                   & \multicolumn{1}{c}{\em kardinal}
\\ \cline{2-4}
& \raisebox{-1.5ex}[1.5ex]{\em subjektiv}  & subjektiver \mbox{ordinaler
Nutzen} & subjektiver \mbox{kardinaler Nuzen}   \\
\cline{2-4}
\raisebox{1.5ex}[-1.5ex]{Vergleichbarkeit}
& \raisebox{-1.5ex}[1.5ex]{\em intersubjektiv}
& intersubjektiver \mbox{ordinaler Nutzen} &
intersubjektiver \mbox{kardinaler Nutzen} \\
\cline{2-4}
\\
\end{tabular}
\end{center}
Spiel- und entscheidungstheoretische Modelle kann man danach einteilen, welche
Art von Nutzen sie voraussetzen. Die empirische Anwendbarkeit solcher Modelle
hängt dann immer davon ab, ob man in einer gegebenen Anwendungssituation das
vorausgesetzte Nutzenkonzept rechtfertigen kann oder nicht (was in der Regel
wiederum eine Frage des Vorhandenseins zuverlässiger Bestimmungsmethoden der
Nutzenwerte des vorausgesetzten Nutzenkonzepts in der gegebenen Anwendungssituation 
ist).

\subsection{Weitere Entscheidungsregeln auf Basis des kardinalen Nutzens}
\subsubsection{Die Optimismus-Pessimismus Regel}

Für die Theorie- und Modellbildung ist der kardinale Nutzen deshalb so
vorteilhaft, weil er es erlaubt, in einem gewissen Rahmen mit Nutzenwerten zu
rechnen. Mit Hilfe des kardinalen Nutzenbegriffs können wir daher nicht nur
endlich guten Gewissens die Minimax-Bedauerns-Regel anwenden, sondern gleich
auch eine ganze Reihe weiterer Entscheidungsregeln erfinden. Eine davon ist die
"`Optimismus-Pessimismus"'-Regel. Diese Regel funktioniert folgendermaßen:
Zunächst legen wir einen Optimismusindex $a$ fest, der zwischen 0 und 1
liegen muss. Dann wählen für jede Handlung (also aus jeder {\em Zeile} der
Entscheidungstabelle) das beste und das schlechteste mögliche Ergebnis
aus. Das beste Ergebnis können wir der Einfachheit halber mit $MAX$
bezeichnen, das schlechteste nennen wir $min$. Nun berechnen wir für jede
Handlung eine Bewertung $R_a$ ("`R"' wie "`rating"') nach folgender Formel:
\marginline{Bewertung mit Hilfe des Optimismusindex}
\begin{displaymath}
R_a = aMAX + (1-a)min
\end{displaymath}
Schließlich wählen wir diejenige Handlung aus, für die $R_a$ am größten ist.
Welche Handlung gewählt wird hängt dabei ganz wesentlich von der Wahl des
Optimismusindex $a$ ab. Aber das ist auch gewollt, denn bei dieser
Entscheidungsregel geht es darum zuerst festzulegen, wie "`optimistisch"' man
sein möchte, und dann auf dieser Grundlage die eigentliche Entscheidung zu
treffen. Die beiden Grenzfälle $a=0$ und $a=1$ entsprechen übrigens
haargenau der letzte Woche besprochenen Maximin ($a=0$) und Maximax-Regel
($a=1$). Die Anwendung der Regel kann an folgendem Beispiel verdeutlicht werden:
\begin{center}
\begin{tabular}{c|c|c|c|}
\multicolumn{1}{c}{}  & \multicolumn{1}{c}{S1}  & \multicolumn{1}{c}{S2}  & 
\multicolumn{1}{c}{S3} 
\\ \cline{2-4}
 A1 & 9 & 1 & 2 \\ \cline{2-4}
 A2 & 5 & 6 & 3 \\ \cline{2-4}
\end{tabular}
\end{center}
Für a = 0.5 ergibt sich:
\begin{eqnarray*}
& R_{A1} = 0.5 \cdot 9 + 0.5 \cdot 1 = 5.0 & \\
& R_{A2} = 0.5 \cdot 6 + 0.5 \cdot 3 = 4.5 & \\
\end{eqnarray*}
Bei einem Optimismus-Index von 0.5 sollte also die Handlung A1 gewählt werden.

Für a = 0.2 ergibt sich dagegen:
\begin{eqnarray*}
& R_{A1} = 0.2 \cdot 9 + 0.8 \cdot 1 = 2.6 & \\
& R_{A2} = 0.2 \cdot 6 + 0.8 \cdot 3 = 3.6 & \\
\end{eqnarray*}
In diesem Fall sollte die Handlung A2 gewählt werden.

Die Handlungsempfehlung, die sich aus der Anwendung
der Optimismus-Pessimismus-Regel ergibt, hängt wie zu erwarten von der Wahl des
Optimismusindex ab. Auch wenn diese Wahl willkürlich ist, stellt sich doch die
Frage, ob es ein Verfahren gibt, um die Wahl wenigstens sinnvoll zu treffen,
oder anders formuliert: Woher weiss ich eigentlich wie optimistisch ich sein
will? Ein Verfahren, das zu Bestimmung des Index vorgeschlagen worden ist, ist
dieses (vgl. \cite[S. 33]{resnik:1987}): 
\marginline{Bestimmung des Optimismusindex} Man nehme die folgende 
einfache Entscheidungstabelle, in welcher in der ersten Zeile die Nutzenwerte 0
und 1 (einer beliebigen kardinalen Nutzenskala) und in der zweiten Zeile in beiden
Spalten ein unbekanntes Ergebnis {\em x} eingetragen worden ist:
\begin{center}
\begin{tabular}{c|c|c|}
\multicolumn{1}{c}{}  & \multicolumn{1}{c}{S1}  & \multicolumn{1}{c}{S2} 
\\ \cline{2-3}
 A1 & 0 & 1 \\ \cline{2-3}
 A2 & x & x \\ \cline{2-3}
\end{tabular}
\end{center}
Dabei soll diesmal die Frage nicht lauten, welche Handlung gewählt werden soll
(um ein möglichst gutes Ergebnis zu erzielen), sondern es soll vielmehr schon
vorgegeben sein, dass wir zwischen den Handlungen A1 und A2 indifferent sind.
Nun müssen wir x genau so groß wählen, dass wir zwischen A1 und A2 tatsächlich
indifferent sind. Haben wir x entsprechend gewählt, dann können wir daraus den
Optimismus-Pessimismusindex ableiten, denn auf Grund der Indifferenz gilt:
\begin{eqnarray*}
R_{A1} & = & R_{A2} \\
a \cdot 1 + (1-a) \cdot 0 & = & a \cdot x + (1-a) \cdot x \\
a & = & x
\end{eqnarray*}
Was ist damit gewonnen? Wir haben auf diese Weise die Wahl des Optimismusindex
aus der Wahl (bzw. Entscheidung im {\em dezisionistischen} Sinne) über die
Indifferenz zwischen zwei Handlungsalternativen abgeleitet. Wenn man annimmt,
dass es leichter ist, anzugeben, ob man zwischen zwei Alternativen indifferent
ist, als die Frage zu beantworten, wie hoch man den eigenen Optimismus auf
einer Skala zwischen 0 und 1 einschätzt, dann vereinfacht das die Wahl des
Optimusmusindex. Wir hätten dann eine Willkürentscheidung auf eine andere
zurückgeführt, die zu treffen uns möglicherweise leichter fällt.

\marginline{Einwände:}
Allerdings wirkt dieses Verfahren etwas gezwungen. Vor allem gibt es einen
gravierenden Einwand:\marginline{1.Risiko\-be\-reit\-schaft ist
situationsspezifisch} Die Frage wie optimistisch oder pessimistisch man
entscheiden sollte, oder, was auf dasselbe hinausläuft, wie risikofreudig oder
risikoavers man sich verhält, dürfte von den meisten Menschen hochgradig
situationsspezifisch beantwortet werden. Insofern erscheint es äußerst
fragwürdig, einen Optimismusindex, den man durch ein abstraktes
Gedankenexperiment bestimmt hat, auf irgendeine konkrete Entscheidungssituation
zu übertragen, der man möglicherweise ein ganz anderes Risikoverhalten zu Grunde
legen möchte. Dann kann man sich das Gedankenexperiment besser gleich sparen und
willkürlich bleibt die Entscheidung über den Optimismusindex ohnehin.

Dieses Willkürelement ist noch aus einem anderen Grund als dem der Schwierigkeit
der Festlegung des Optimismusindex problematisch:
\marginline{2.Nachträgliche "`Rationalisierung"' von Entscheidungen}
Wenn eine Entscheidungsregel derartige Willkürelemente enthält, dann lädt sie
geradezu dazu ein, zuerst die Entscheidung vollkommen intuitiv zu treffen, und
sie erst im Nachhinein durch die Wahl eines geeigneten Index zu
"`rationalisieren"'. Das könnte besonders dann problematisch werden, wenn die
entscheidungtreffenden Personen anderen für ihre Entscheidung
rechenschaftspflichtig sind, denn es lässt sich dann nicht mehr nachvollziehen,
ob die Entscheidung tatsächlich "`verantwortlich"' getroffen wurde.

Daneben ist die Optimismus-Pessimismus-Regel mit ähnlichen Schwierigkeiten
behaftet, wie die Maximin und die Minimax-Bedauerns-Regel. 
\marginline{3.Ver\-nach\-läs\-si\-gung mittlerer Optionen}
Da sie jeweils nur
zwei Werte jeder Zeile in das Kalkül einbezieht, lassen sich leicht Fälle
konstruieren, in denen sie unplausibel erscheint:
\begin{center}
\begin{tabular}{c|c|c|c|c|c|c|}
\multicolumn{1}{c}{ } & \multicolumn{1}{c}{$S_1$} &
\multicolumn{1}{c}{$S_2$} & \multicolumn{1}{c}{$S_3$} & 
\multicolumn{1}{c}{\ldots} & 
\multicolumn{1}{c}{$S_{99}$} & \multicolumn{1}{c}{$S_{100}$} \\ \cline{2-7}
$A_1$ & 2 & 1 & 1 & $\cdots$ & 1 & 0 \\ \cline{2-7}
$A_2$ & 2 & 0 & 0 & $\cdots$ & 0 & 0 \\ \cline{2-7}
\end{tabular}
\end{center}

In diesem Fall würde die Optimismus-Pessimismus-Regel immer zur Indifferenz
zwischen beiden Handlungen führen, obwohl intuitiv die Handlung A1 sicherlich
als die bessere beurteilt werden müsste. 

Schließlich existiert noch ein weiterer Einwand, der auf einer etwas
raffinierteren Konstruktion beruht, nämlich auf der sogenannten
"`Mischungsbedingung"' ({\em mixture-condition}),
\marginline{Mischungs\-be\-ding\-ung}
 die -- leicht vereinfacht --
besagt: Wenn eine Person indifferent zwischen zwei Handlungsalternativen ist,
dann ist sie auch indifferent zwischen diesen beiden Handlungen und einer dritten
Handlung, die darin besteht, eine Münze zu werfen und bei "`Kopf"' die erste
Handlung und bei "`Zahl"' die zweite Handlung zu wählen. Betrachten wir die
folgende Tabelle:

\begin{center}
\begin{tabular}{c|c|c|}
\multicolumn{1}{c}{}  & \multicolumn{1}{c}{S1}  & \multicolumn{1}{c}{S2} 
\\ \cline{2-3}
 A1 & 0 & 1 \\ \cline{2-3}
 A2 & 1 & 0 \\ \cline{2-3}
\end{tabular}
\end{center}

Nach der Optimismus-Pessimismus-Regel herrscht zwischen beiden
Handlunsalternativen völlige Indifferenz, und zwar unabhängig von der Wahl des
Optimismusindex {\em a}. Fügt man nun die Münzwurfalternative hinzu, dann
ergibt sich folgende Entscheidungstabelle:\footnote{Bei der Nutzenbewertung der
Ergebnisse der Münzwurfhandlung wurde implizit bereits die {\em
Erwartungsnutzenhpyothese} zugrunde gelegt, die besagt, dass der
Erwartungsnutzen gleich dem erwarteten Nutzen multipliziert mit der
Eintrittswahrscheinlichkeit ist. Strenggenommen kann auch das Ergebnis der
Münzwurfhandlung nur 0 oder 1 sein.}

\begin{center}
\begin{tabular}{c|c|c|}
\multicolumn{1}{c}{}  & \multicolumn{1}{c}{S1}  & \multicolumn{1}{c}{S2} 
\\ \cline{2-3}
 A1 & 0 & 1 \\ \cline{2-3}
 A2 & 1 & 0 \\ \cline{2-3}
 A3 & $\frac{1}{2}$ & $\frac{1}{2}$ \\ \cline{2-3}
\end{tabular}
\end{center}

Angenommen der Optimismus-Pessimismus-Index wäre $a = \frac{2}{3}$. Dann ergibt
sich daraus:
\marginline{4.Verletzung der "`Mischungs\-bedingung"'}
\begin{eqnarray*}
R_{A1} & = \frac{1}{3} \cdot 0 + \frac{2}{3} \cdot 1 = & 2/3 \\
R_{A2} & = \frac{2}{3} \cdot 1 + \frac{1}{3} \cdot 0 = & 2/3 \\
R_{A3} & = \frac{2}{3} \cdot \frac{1}{2} + \frac{1}{3} \cdot \frac{1}{2} =
\frac{1}{2}
\end{eqnarray*}
Nach der Optimismus-Pessimismus-Regel müssten die Handlungen A1 und A2 der
"`Münzwurfalternative"' vorgezogen werden, unter Verletzung der
Mischungsbedingung. Die Mischungsbedingung lässt sich nur erfüllen, wenn das
beste und das schlechteste mögliche Ergebnis genau gleich gewichtet
werden, d.h. bei einem Optimismusindex von $a=\frac{1}{2}$. 

\marginline{Einwände gegen die Mischungsbedingung}
Wie auch bei den denkbaren Einwänden gegen die anderen Entscheidungsregeln, lässt
sich darüber streiten, ob die Verletzung der "`Mischungsbedingung"' ein Nachteil
oder, eher im Gegenteil, eine besondere Eigenschaft der
Optimismus-Pessimismus-Regel ist. ("`It's not a bug, it's a feature!"') Wenn
jemand optimistisch ist, dann besagt das ja gerade, dass die Person eher geneigt
ist, an den Erfolg zu glauben als an eine 50:50 Chance von Erfolg und Misserfolg,
so dass es nicht verwunderlich ist, dass sie eine Handlung, an deren Erfolg sie
glaubt, einem Münzwurf vorzieht, von dem sie weiß, dass die Chancen
gleichverteilt sind. Widersprüchlich wäre das optimistische (oder pessimistische)
Verhalten bei der gegebenen Entscheidungstabelle aber immer noch insofern, als
die Person eigentlich nur {\em entweder} an den mehr als 50\%-igen Erfolg von S1
{\em oder} von S2 glauben dürfte, aber -- sofern die Zustände S1 und S2 von den
Handlungen unabhängig sind -- nicht daran, dass sie in jedem Fall die höheren
Erfolgschancen hat.

\subsubsection{Das Prinzip der Indifferenz}
\label{Indifferenzprinzip}
Wenn wir die Nutzenwerte als kardinale Nutzenwerte interpretieren und daher mit
ihnen rechnen dürfen, wie das bei der Optimismus-Pessimismus-Regel der Fall
ist, dann besteht eine der naheliegendsten Arten, die unterschiedlichen
Handlungsalternativen in eine Rangordnung zu überführen, darin, einfach alle
Zahlen in jeder Zeile aufzusummieren und die Handlungsalternative mit der
höchsten Zeilensumme zu wählen. An einem Beispiel betrachtet sieht das
Verfahren folgendermaßen aus:

\begin{center}
\begin{tabular}{c|c|c|c|c|c|cc}
\multicolumn{1}{c}{ } & \multicolumn{1}{c}{$S_1$} &
\multicolumn{1}{c}{$S_2$} & \multicolumn{1}{c}{$S_3$} & 
\multicolumn{1}{c}{$S_4$} & 
\multicolumn{1}{c}{$S_5$} & \multicolumn{1}{c}{ $\sum$ } \\ \cline{2-6}
$A_1$ & 8  & 2  & -7 & 3  & 3 &  9 \\ \cline{2-6}
$A_2$ & -5 & -3 & 5  & 12 & 4 & 13  \\ \cline{2-6}
\end{tabular}
\end{center}

In diesem Fall würde also die Handlung A2 gewählt werden, weil die Summe der
erzielbaren Nutzenwerte größer ist als bei der Handlung A1. Werden die
Nutzenwerte einer Zeile einfach aufsummiert, dann bedeutet das, dass sie alle
gleich gewichtet werden. Dem Summierungsverfahren liegt damit implizit ein
Prinzip zu Grunde, das man auch als das {\em Prinzip der Indifferenz} bezeichnet.
\marginline{Prinzip der Indifferenz} Es besagt, dass wir alle Ereignisse als
gleichwahrscheinlich betrachten sollten, solange wir nicht wissen, mit welcher
Wahrscheinlichkeit eines von mehreren Ereignissen eintreten wird.\footnote{In
der Fachliteratur wird statt vom "`Prinzip der Indifferenz"' zuweilen auch vom
"`Prinzip des (un-)zureichenden Grundes"' gesprochen \cite[S. 35ff]{resnik:1987}.
Beim "`Prinzip des (un-)zureichenden Grundes"' handelt es sich aber um einen
allgemeineren philosophischen Gedanken, der in der Philosophiegeschichte immer
wieder in unterschiedlichen Ausprägungen und Formulierungen aufgetreten ist. In
der einfachsten Form besagt es, dass nichts ohne Ursache geschieht. Man kann es
auch so auffassen, dass in einer Reihe von gleichartigen Ereignissen keine
Ausnahmen auftreten können, ohne dass es dafür einen zureichenden Grund gibt,
d.h. der Ausnahmefall muss sich in irgendeiner qualitativen Hinsicht von den
anderen Fällen unterscheiden. Das Prinzip des unzureichenden Grundes ist ein {\em
heuristischer Grundsatz} (ein Hilfsmittel unserer Erkenntnis). Ontologische, d.h.
die Natur der Gegenstände selbst bzw. das Wesen des Seins betreffende Bedeutung
kommt ihm wenn überhaupt nur in einem deterministischen Universum zu (Vgl.
\cite[S. 130]{schurz:2006}). Das hier besprochene "`Prinzip der Indifferenz"'
kann man vage auf das Prinzip des (un-)zureichenden Grundes zurückführen.}


In diesem Zusammenhang ist noch einmal darauf hinzuweisen, dass ein subtiler
Unterschied zwischen dem vom "`Prinzip der Indifferenz"' erfassten Fall besteht,
in dem wir nicht wissen, mit welcher Wahrscheinlichkeit ein bestimmtes Ereignis
eintritt ("`{\em Unwissen}"'), und dem vergleichsweise "`harmloseren"' Fall, in
dem wir bloß nicht wissen, welches Ereignis eintritt, aber über die
Wahrscheinlichkeiten der Ereignisse auf Grund unserer Kenntnis des empirischen
Vorgangs, um den es geht, genaue Aussagen machen können ("`{\em Risiko}"'). Beim
Würfeln oder bei einem Münzwurf etwa wissen wir auf Grund unser Kenntnis von
Würfeln und Münzen, dass die verschiedenen möglichen Ereignisse gleichverteilt
sind. Die Rechtfertigung dafür, dass wir beim Würfeln oder auch beim Werfen einer
Münze von einer Gleichverteilung ausgehen, ergibt sich aus dieser Kenntnis. Dem
Prinzip der Indifferenz liegt keine vergleichbare Rechtfertigung zu Grunde. Es
handelt sich um ein philosophisches oder, wenn man so will, sogar {\em
metaphysisches Postulat}, dessen Annahme keinesfalls zwingend ist (wohingegen die
Annahme der Gleichverteilung von Würfelergebnissen oder Münzwürfen genauso
zwingend ist, wie andere Aspekte der alltäglichen physischen Wirklichkeit, wie
etwa, dass "`morgens die Sonne aufgeht"', dass "`dort eine Wand steht"' etc.).

Die auf dem Prinzip der Indifferenz beruhende Entscheidungsregel hat die
Eigenschaft (wenn man so will: den Vorzug), dass sie sowohl die {\em
Mischungsbedingung} erfüllt als auch {\em Unabhängigkeit von irrelevanten
Alternativen} garantiert und selbstverständlich weiterhin {\em dominierte
Alternativen} ausschließt. Trotzdem wird man in bestimmten Situation, z.B. in
Situationen, in denen es vor allem darum geht, Schaden zu begrenzen, auf andere
Entscheidungsregeln wie die Maximin-Regel zurückgreifen. Unter "`Unwissen"' gibt
es viele je nach Situation mehr oder weniger gute Entscheidungsregeln, aber keine
eindeutig beste Regel.

\subsubsection{Paradoxien des Indifferenzprinzips}
\label{IndifferenzPrinzipParadoxien}

Einwände gegen das Indifferenzprinzip werden häufig daraus abgeleitet, dass sich
bei der Anwendung des Prinzips unter bestimmten Bedingungen Paradoxien ergeben.
Was es damit auf sich hat, und ob diese Paradoxien ein Problem bei der Anwendung
des Indifferenzprinzips bei den hier besprochenen Entscheidungen unter
Unwissenheit darstellen, soll nun kurz erörtert werden.\footnote{Neuerlich hat
Rudolfo Cristofaro den Anspruch erhoben, das Indifferenzprinzip in einer Form
gefasst zu haben, in der keine Paradoxien mehr entstehen
\cite[]{cristofaro:2008}. Er geht nicht unmittelbar darauf ein, wie mit seiner
Neuformulierung des Prinzips die Paradoxien umgangen werden. Seine Ausführungen
legen die Vermutung nahe, dass dies nur dadurch ermöglicht wird, dass er
verlangt, dass die Informationen über das "`experimentelle Design"' mit in die
Situationsbeschreibung einfließen müssen. Eine rein logische Rechtfertigung des
Indifferenzprinzips wäre damit nicht gegeben. Seine Lösung ginge dann -- bis
evtl. auf die allgemeinere Formulierung -- nicht fundamental über bestehende
Lösungen hinaus.} Um diese Paradoxien zu erläutern, muss schon ein wenig auf die
Wahrscheinlichkeitsrechnung vorgegriffen werden (Kapitel
\ref{Wahrscheinlichkeitsrechnung}). Es genügt allerdings zu wissen, dass die
Wahrscheinlichkeit eines Ereignisses immer eine reelle Zahl von 0 bis 1 ist, und
dass sich die Wahrscheinlichkeiten einer Reihe von Alternativen, die einander
ausschließen, von denen aber irgendeine auf jeden Fall eintreten muss, zu 1
aufaddieren, und dass man die Wahrscheinlichkeit eines Ereignisses üblicherweise
mit $P(Ereignis)$ darstellt.

{\em Buch-Paradox:}\marginline{Buch-\-Paradox}\label{BuchParadox} Das erste
Paradoxon entsteht folgendermaßen: In der Uni-Bibliothek steht eine Ausgabe von
Schopenhauers "`Die Welt als Wille und Vorstellung"'. Wenn jemand noch nicht in
der Bibliothek war, dann weiß sie oder er nicht, ob das Buch einen blauen oder
keinen blauen Umschlag hat. Nach dem Indifferenzprinzip müsste die
Wahrscheinlichkeit dafür, dass das Buch einen blauen Umschlag hat $P(blau)$ also
1/2 betragen. Aber mit genau demselben Argument kann gefolgert werden, dass
$P(rot)$, $P(gelb)$, $P(lila)$ etc. alle den Wert 1/2 haben. Damit haben wir eine
Reihe von sich wechselseitig ausschließenden Alternativen, deren
Wahrscheinlichkeiten sich auf eine Zahl größer 1 aufaddieren, was wiederum der
Definition der Wahrscheinlichkeit widerspricht \cite[S. 37f.]{gillies:2000}.

Wie könnte man das Paradox lösen? Denkbar wäre folgender Lösungsansatz:
Wahrscheinlichkeiten dürfen nur unteilbaren {\em Elementar-}ereignissen
zugewiesen werden. Soll heißen: Bevor man das Prinzip der Indifferenz anwendet
ist zunächst sicherzustellen, dass sämtliche Ereignisse, auf die man es anwendet
(vorbehaltlich unseres Wissens darüber) unteilbare Elementarereignisse sind. Im
Fall des Buch-Paradoxons ist das Ereignis "`nicht blau"' offenbar kein
Elementarereignis, da wir wissen, dass noch andere Farben in Frage kommen. Nun
stellt sich aber das weitere Problem, dass wir gar nicht wissen, wie viele andere
Farben in Frage kommen. Der Lösungsansatz beinhaltet also, dass wir das
Indifferenzprinzip überhaupt nur dann anwenden können, wenn wir zumindest die
Menge der Elementarereignisse kennen. Ist das aber der Fall, so hilft uns das
Indifferenzprinzip immerhin noch dabei, diesen Elementarereignissen in sinnvoller
Weise Wahrscheinlichkeiten zuzuweisen, wenn uns deren objektive
Wahrscheinlichkeiten unbekannt sind.

{\em Wasser-Wein-Paradox:}\marginline{Wasser-Wein-Paradox}
\label{WasserWeinParadox} Leider funktioniert dieser Lösungsansatz nicht mehr bei
den sogennanten "`geometrischen Wahrscheinlichkeiten"', bei denen wir es statt
mit diskreten (d.h. zählbaren) mit kontinuierlichen Größen zu tun haben, wie uns
das Wasser-Wein-Paradox vor Augen führt. Bei diesem Paradox geht es um Folgendes
\cite[p. 84]{howson:2000}: Angenommen wir haben eine Mischung von Wasser und
Wein, von der wir wissen, dass das Verhältnis von Wasser zu Wein bei dieser
Mischung irgendwo zwischen "`halbe-halbe"' und "`doppelt soviel Wasser wie Wein"'
liegt. Die unbekannte Menge des Wassers $x$ liegt bezogen auf die gegebene Menge
von Wein also irgendwo zwischen $1$ und $2$ (die Grenzen eingeschlossen). Und
umgekehrt liegt der Mengenanteil des Weins $w$ im Verhältnis zum Wasser irgendwo
zwischen $\frac{1}{2}$ und $1$. Nach dem Indifferenzprinzip sollte die
Wahrscheinlichkeit, dass die Wassermenge $x$ zwischen $1$ und $\frac{3}{2}$
liegt, sicherlich genauso groß sein, wie die Wahrscheinlichekit, dass sie
zwischen $\frac{3}{2}$ und $2$ liegt, also jeweils $1/2$. Da der Weinanteil genau
im umgekehrten Verhältnis zum Wasseranteil steht, also $w = 1/x$, so ergibt sich
daraus, dass die Wahrscheinlichkeit, dass $w$ zwischen $\frac{1}{2}$ und
$\frac{2}{3}$ bzw. zwischen $\frac{2}{3}$ und $1$ liegt, ebenfalls jeweils $1/2$
betragen muss. Das ist aber mit dem Prinzip der Indifferenz unvereinbar, das ja
fordert, dass die Wahrscheinlichkeit für gleich große Intervalle gleich groß sein
muss.

Die Lösung dieses Paradoxons ist deshalb schwieriger als die des
Buch-Paradoxons, weil die beiden Größen, die hier involviert sind, die relative
Menge des Wassers $x$ und die relative Menge des Weins $w$ sich anders als
"`blau"' und "`nicht blau"' vollkommen symmetrisch verhalten. Trotzdem ist
eine Lösung denkbar, indem man die relativen Mengenangaben durch absolute
Mengenangaben ersetzt. Beziehen wir die Wein- und die Wassermenge auf eine
konstante Grundmenge von 6 Mengeneinheiten, dann liegt die Weinmenge zwischen 2
und 3 Mengeneinheiten und die Wassermenge zwischen 3 und 4 Mengeneinheiten. Die
Schwankungsbreite betrifft dann sowohl für Wein als auch für Wasser ein
Intervall von genau einer Mengeneinheit, so dass die Anwendung des
Indifferenzprinzips wahlweise auf Wein oder auf Wasser zu keinen Widersprüchen
mehr führen kann. 

Diese Lösung des Paradoxons setzt allerdings ebenso wie die vorhergehende ein
ontologisches Wissen über die Situation voraus, in der wir das Indifferenzprinzip
anwenden. Dieses Wissen geht über die bloße Kenntnis der Anzahl der involvierten
Parameter (zwei, nämlich $x$ und $w$), ihres möglichen Wertebereichs ($[1, 2]$,
$[\frac{1}{2}, 1]$) und ihrer wechselseitigen Beziehung $w = 1/x$ hinaus.
Insofern ist die gefundene Lösung nicht mathematisch verallgemeinerbar. Wenn wir
nur die rein mathematischen Beziehungen zwischen den beteiligten Größen
betrachten, dann stehen wir -- etwas vereinfacht betrachtet -- vor dem Problem,
dass wir das Prinzip der Indifferenz nicht gleichzeitig auf das Intervall $[a,b]$
anwenden können (indem wir gleichgroßen Teilintervallen gleichgroße
Wahrscheinlichkeiten zuweisen) und auf das Intervall $[\frac{1}{a},
\frac{1}{b}]$. Haben wir das Prinzip der Indifferenz schon auf das erste
Intervall angewendet, dann haben wir automatisch eine Entscheidung damit
getroffen, es nicht auf das zweite Intervall anzuwenden und umgekehrt. Rein
mathematisch betrachtet, können wir aber gar nicht unterscheiden, ob $a$ und $b$
oder ob $\frac{1}{a}$ und $\frac{1}{b}$ die Basisgrößen sind, von denen wir
auszugehen haben.\footnote{Man kann das Problem nicht durch dern Vorschlag lösen,
dass man die Entscheidung zwischen den beiden Alternativen $[a,b]$ und
$[1/a,1/b]$ mangels besserem Wissen nach belieben treffen darf, denn da statt
$1/x$ jede beliebige mathematische Transformation stehen könnte, hieße dies, dass
man bezüglich der Wahrscheinlichkeitsverteilung von $[a,b]$ jede beliebige Wahl
treffen darf, was aber gerade das Gegenteil dessen ist, was mit dem Prinzip der
Indifferenz beabsichtigt wird!} Und auch empirische Größen bieten dafür nicht
zwangsläufig hinreichende Anhaltspunkte. Man denke etwa an Lichtwellen, die wir
durch ihre Wellenlänge $\lambda$ oder ihre Frequenz $f$ angeben können, wobei
beide in dem Verhältnis $\lambda = 1/f$ zueinander stehen, ohne dass man eine der
beiden Angaben in irgendeiner Weise als privilegiert auszeichnen könnte. Das
bedeutet aber, dass wir das Indifferenzprinzip ohne die Gefahr eines Paradoxons
nur heranziehen können, wenn die Anwendungssituation das zulässt und wir über ein
ausreichendes Hintergrundwissen darüber verfügen. Bei völligem Unwissen hilft es
nicht weiter.

Inwiefern sind die hier besprochenen Paradoxien ein Problem für die Anwendung des
Indifferenzprinzips auf Entscheidungen unter Unwissenheit?
\marginline{Anwendbarkeit des Indifferenzprinzips} Hier sind zwei Situationen zu
unterscheiden:

\begin{enumerate}
 
  \item Wir verfügen über ein hinreichendes Hintergrundwissen der Situation,
  dass es uns erlaubt, das Indifferenzprinzip eindeutig auf die Situation
  anzuwenden. (Z.B. müssten wir beim Buch-Paradoxon die Menge der in Frage
  kommenden Farben kennen.) Dann dürfen wir das Indifferenzprinzip anwenden,
  sollten uns aber bewusst sein, dass die Annahme jeder anderen
  Wahrscheinlichkeitsverteilung als der Gleichverteilung genauso legitim wäre.
  Aber wir könnten wenigstens sicher sein, dass die Anwendung dieses Prinzips
  nicht zu Entscheidungsempfehlungen führt, die sich in kontingenter
  Weise wandeln, wenn
  wir die Zustandsbeschreibungen durch äquivalente andere
  Zustandsbeschreibungen austauschen.

  \item Wir verfügen nicht über ein entsprechendes Hintergrundwissen. Dann
  können wir das Prinzip nicht anwenden, denn es liefert für dieselbe
  Entscheidungssituation widersprechende Empfehlungen. 
 
\end{enumerate}



