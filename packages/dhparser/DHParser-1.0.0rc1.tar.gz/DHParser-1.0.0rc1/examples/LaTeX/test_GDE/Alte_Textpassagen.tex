% aus Vorlesung2.tex:

Ein wesentliches Problem der lexikalischen Maximin-Regel besteht darin, dass
sie dominierte Alternativen nicht immer ausschlie�t. Man betrachte dazu
folgende Tabelle:

\begin{center}
\begin{tabular}{l|c|c|c|}
\multicolumn{1}{c}{ } & \multicolumn{1}{c}{$S_1$} &
\multicolumn{1}{c}{$S_2$} & \multicolumn{1}{c}{$S_3$}
\\ \cline{2-4}
$A_1$   &    -1  &   2 &  100  \\ \cline{2-4}
$A_2$   &    -1  &  -1 &   3   \\ \cline{2-4}
\end{tabular}
\end{center}

In diesem Fall wird die Handlung $A_2$ durch die Handlung $A_1$ dominiert. Sie
ist also unter allen Umst�nden schlechter als die Handlung $A_1$. Geht man
nach der lexikalischen Maximin-Regel vor, so w�rde man im ersten Schritt keine
Entscheidung treffen k�nnen, weil f�r beide Handlungsalternativen das minimale
Resultat mit -1 gleichgro� ist. Streicht (oder ignoriert) man im n�chsten die
schlechtesten Resultate, um sich nach den zweitschlechtesten zu richten, so
m�sste man die dominierte Handlung $A_2$ w�hlen, da der zweitschlechteste Wert
von $A_2$ (u=3) gr��er ist als der zweitschlechteste Wert von $A_1$ (u=2)!

Man kann dem Problem dadurch begegnen, dass man festlegt, dass der schlechteste
Wert, sofern er in einer Zeile mehrmals vorkommt, nur einmal gestrichen werden
sollte. In welcher Spalte er gestrichen wird w�re in dem Fall sogar gleichg�ltig,
da man auch in der Folge die Werte ja niemals spaltenweise, sondern immer nur die
jeweils schlechtesten Werte jeder Zeile miteinander
vergleicht. Zudem w�re die
Streichung nur eines Wertes dadurch motiviert, dass ein h�ufigeres Vorkommen des
schlechtesten Wertes gegen�ber einem nur einfachen Vorkommen als Nachteil einer
Handlungsalternative aufgefasst werden sollte.

Eine andere M�glichkeit, dass Problem zu l�sen, besteht darin, die
lexikalische Maximin-Regel mit der Dominanz-Regel dergestalt zu kombinieren,
dass man in jedem Schritt als erstes alle dominierten Handlungsalternativen
streicht und dann die Maximin-Regel anwendet. Das Auftreten des oben
beschrieben "`Problems"' ist so unm�glich gemacht, denn entweder gab es (in
einem gegebenen Anwendungsschritt) das Problem nicht, weil keine
Handlungsalternative dominiert wurde, oder die dominierten
Handlungsalternativen werden gestrichen, so dass das Problem nicht mehr
vorhanden ist, sobald die Maximin-Regel angewendet wird.