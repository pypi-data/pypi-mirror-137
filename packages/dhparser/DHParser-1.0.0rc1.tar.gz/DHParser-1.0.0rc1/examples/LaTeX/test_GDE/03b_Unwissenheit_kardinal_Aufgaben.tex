\subsection{Aufgaben}
\begin{enumerate}

\item Welche der folgenden Transformationen sind {\em ordinale} und welche {\em
positive lineare} Transformationen (und welche keins von beiden)? (Es sei
angenommen, dass x eine beliebige reelle Zahl sein kann):
   \begin{enumerate}
     \item t(x) = $x - 5$
     \item t(x) = $x^2$
     \item t(x) = $x^3$
     % \item t(x) = $(2x+5)^2 - 12 + 6x - 4x^2$
     \item t(x) = $3x + 5 - 4x$
   \end{enumerate}
   
\item Quizfrage: Ist die Fahrenheitskala zur Messung der Temperatur eine
Intervallskala oder eine Verhältnisskala?
   
\item Kann es bei Verwendung der Optimismus-Pessimismus-Regel dazu kommen, dass
dominierte Handlungen gewählt werden?

\item Die Optimismus-Pessimismus-Regel hat die Schwäche, dass immer nur zwei
Einträge jeder Zeile (das Maximum und da Minimum) der Entscheidungstabelle
berücksichtigt werden. Denken Sie sich eine Verbesserung der
Optimismus-Pessimismus-Regel aus, die alle Einträge einer Zeile berücksichtigt.

\item Zeige, dass das Prinzip des unzureichenden Grundes niemals eine dominierte
Handlungsalternative empfiehlt.
 
\item Erkläre (möglichst anhand eines Beipiels), warum das Prinzip der
Indifferenz den kardinalen Nutzen voraussetzt.

\item Bei zwei Münzwürfen gibt es drei Möglichkeiten: a) 2-mal Kopf b) 1-mal
Kopf und 1-mal Zahl c) 2-mal Zahl. Jemand schließt daraus, dass man nach dem
{\em Prinzip der Indifferenz} jeder dieser Möglichkeiten die Wahrscheinlichkeit
$1/3$ zuweisen muss. Warum ist das falsch und was sind die richtigen
Wahrscheinlichkeiten?

\item \label{AufgabeDritteAlternative} Worin unterscheiden sich die Beispiele
für die mögliche Relevanz dritter Alternativen auf Seite \pageref{dritteAlternativen}? 
Weshalb ist in den 
unterschiedlichen Beispielen die dritte Alternative jeweils "`relevant"'?
Lassen sich einzelne der Beispiele durch eine entsprechende Interpretation der
Ausgangssituation (sprich "`Problemspezifikation"') doch noch so mit der Theorie
vereinbaren, dass das Prinzip der Unabhängigkeit von dritten Alternativen nicht
verletzt werden müsste.


~\\{\bf schwierigere Aufgaben}\\

\item Zeige: Wenn man eine Entscheidungstabelle positiv linear in eine andere
überführt, dann ist auch die zugehörige Bedauernstabelle eine positiv linear
transformierte (genaugenommen sogar ein positives Vielfaches, warum?) der
ursprünglichen Bedauernstabelle. (Was müsste man von der Minimax-Bedauernsregel
halten, wenn das nicht der Fall wäre?)

\item Zeige: Positiv lineare Transformationen sind transitiv, d.h. wenn die
Skala u' durch positiv lineare Transformation aus der Skala u hervorgeht und
Skala u' durch eine (nicht notwendigerweise dieselbe) positiv lineare
Transformation in u'' überführt werden kann, dann kann gibt es auch eine
positiv lineare Transformation, die u unmittelbar in u'' überführt. Warum ist
diese Eigenschaft wichtig?

% \item Als Vorgeschmack eine Aufgabe zu bedingten Wahrscheinlichkeiten: In einer
% Fernsehshow muss die Kandidatin eine von drei verschlossenen Türen wählen. Hinter 
% einer der Türen befindet sich als Hauptpreis ein Auto. Wählt
% die Kandidatin die richtige Tür, so bekommt sie das Auto, sonst nichts. Um die
% Sache spannender zu machen, wird nach folgenden Regeln gespielt. Erst darf sich
% die Kandidatin eine Tür aussuchen, dann öffnet der Showmaster, der hinter
% welcher Tür das Auto steht, eine von den verbleibenden Türen, wobei er
% natürlich niemals die Tür schon öffnet, hinter der sich das Auto befindet. Nun
% bekommt die Kandidatin die Gelegenheit entweder bei der zuerst gewählten Tür zu
% bleiben, oder sich noch einmal umzuentscheiden, und die andere noch
% geschlossene Tür zu wählen. Wie sollte sie sich entscheiden, um ihre
% Gewinnchancen zu maximieren?

\end{enumerate}

