\section{Wissenschaftskritische Diskussion der Reichweite und Grenzen der formalen Entscheidungstheorie in der Philosophie}

{\em Späterer Zusatz (2019).} An vielen Stellen in dieser Vorelsung wurde
bereits auf die Fragwürdigkeiten und Schwächen der Entscheidungstheorie
hingewiesen. Diese Schwächen fallen ganz besonders dann ins Gewicht, wenn man
die Entscheidungstheorie nicht als eine Hilfsmittel betrachtet wird, mit dem man
Anhaltspunkte zur Lösung einer beschränkten Klasse von Entscheidungsproblemen
z.B. aus dem Bereich der Betriebswirtschaft gewinnen kann, sondern als eine
universale Logik des menschlichen Handelns, wie das in der analytischen
Philosophie gerne getan wird. Im folgenden sollen - nach einer kleinen bösen,
aber notwendigen Vorrede über die analytische Philosophie - zusammenfassend die
bestehenden erheblichen Defizite der formalen Entsscheidungstheorie betrachtet
werden.

\subsection{Ein paar warme Worte zur Zerstörung der analytischen Philosophie}

In der analytischen Philosophie gilt der stillschweigende Grundsatz,
dass ein Thema nur dann Gegenstand legitimer philosophischer
Diskussion ist, wenn die Diskussion mit formalen (mathematischen)
Methoden geführt wird, oder, wie man auch sagen könnte, wenn man einen
formalwissenschaftlichen Vorwand dafür findet, das Thema zu
diskutieren.  Unglücklicherweise sind viele der erfolgversprechendsten
Ansätze zum Verständnis und zur Erklärung menschlichen Handelns
überhaupt nicht formal. (Kleine Hausaufgabe für die Leserinnen und
Leser: Woran könnte das liegen? Sind bloß die gegenwärtigen formalen
Ansätze zur Erklärung menschlichen Verhaltens hoffnungslos
unzureichend, oder gibt es da prinzipielle Grenzen?) Die analytische
Philosophie ist aber auf Grund einer willentlichen und mittlerweile
durch das System der analytischen Philosophie (d.h. derjenigen
institutionellen und habituellen Faktoren, die für die Karriere eines
Philosophen oder einer Philosophin innerhalb der analytischen
Philosophie entscheidend sind) verfestigen Grundentscheidung nicht
mehr in der Lage, nicht formale Theorien adäquat zu rezipieren und
muss sich zwangsläufig auf diejenigen formalen Theorien menschlichen
Handelns beschränken, die da sind - seien sie auch noch so
erklärungsuntauglich. Und besonders die formale Entscheidungs- und
Spieltheorie ist mit dem Anspruch, eine universale Theorie
menschlichen Handelns zu liefern, hoffnungslos überfordert.

Das Gegenargument lautet: "Aber kein(e) analytische(r) Philosoph/in
hat jemals diese Ansicht vertreten! Vielmehr konzentriert sie sich
lediglich aus Gründen thematischer Beschränkung auf die formalen
Theorien menschlichen Handelns. Damit ist keineswegs die Behauptung
verbunden, das andere, nicht formale Theorien zur Erklärung
menschlichen Handelns nicht geeignet sein könnten." Doch dieses
Gegenargument ist erstens unehrlich, denn auch wenn niemand behauptet,
man müsse sich zur Erklärung menschlichen Handelns auf formale
Theorien beschränken, so verhalten sich doch fast alle analytischen
Philosophen und Philosophinnen so als müsse man es tun. Sie
ignorieren konsequent alle anderen Ansätze und sie würden
innerhalb ihrer eigenen Horde einiges Befremden hervorrufen, wenn sie
nicht formale Ansätze -- und sei es auch nur vergleichsweise --
einbeziehen würden.  Auch ich habe das ja in dieser Vorlesung nicht
getan. Mea culpa - aber das war leider mein Job. Aber wenigstens
verheimliche ich Euch nicht, dass die formale Spiel- und
Entscheidungstheorie Mist ist (dazu gleich mehr). Und zweitens, kann
man eine Theorie -- auch unter dem Vorbehalt thematischer Beschränkung
-- kaum sinnvoll bewerten, wenn man nicht auch mögliche Alternativen
zur ihr in Betracht zieht. Thematische Beschränkung kann und darf
keine Entschuldigung für mutwillige Ignoranz sein. Die Grenzen der
eigenen Kräfte sind schon eher eine Entschuldigung - aber das heisst
nur, dass man evtl. arbeitsteilig an die Sache herangehen muss.

\subsection{Die drei zentralen Schwächen der formalen Entscheidugnstheorie}

Aber warum taugt die Entscheidungstheorie nichts? Dafür gibt es im wesentlichen
drei intrinsiche Gründe:

\begin{enumerate}


\subsubsection{Unrealistisch}

vollständig geordnete und transitive Präferenzen


\subsubsection{Partiell selbstwidersprüchlich}

Der Satz von Arrow angewandt auf multikriterielle Entscheidungsprobleme


\subsubsection{Keine messbaren Größen}

weder kardinale noch ordinale Nutzenwerte sind messbar


\subsection{Relevanz}
