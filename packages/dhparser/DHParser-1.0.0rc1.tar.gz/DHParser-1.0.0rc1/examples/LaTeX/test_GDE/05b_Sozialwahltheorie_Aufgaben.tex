\subsection{Aufgaben}

\begin{enumerate}
  \item \label{AufgPL0} Gegeben seien drei Individuen $A$, $B$, $C$ und drei
  Alternativen $x$, $y$, $z$. Die Präferenzen seien folgendermaßen verteilt:
  \begin{center}
  \begin{tabular}{ccc}
  $A$ & $B$ & $C$ \\
  \cline{1-3}
  $z$ & $x$ & $y$ \\
  $x$ & $y$ & $z$ \\
  $y$ & $z$ & $x$ \\
  \end{tabular}
  \end{center}
  Angenommen, um die kollektive Entscheidung zu treffen, welche Alternative
  gewählt werden soll, sind paarweise Stichwahlen vereinbart worden, und $A$
  ist zum Wahlleiter ernannt worden, mit dem Recht die Reihenfolge festzulegen, 
  in der über jeweils zwei Alternativen abgestimmt worden ist. 
  
  \begin{enumerate}
    \item In welcher Reihenfolge sollte $A$ abstimmen lassen, damit die von $A$
    bevorzugte Alternative $z$ mit Sicherheit gewinnt?
    \item Angenommen, $B$ bemerkt $A$s Plan. Kann $B$ durch "`strategisches
    Wählen"' den Plan von $A$ durchkreuzen? Wenn ja, wie?
  \end{enumerate}
  
  \item Verständnisfrage: Was ist der Unterschied zwischen der "`kollektiven
  Präferenz"' und der "`Präferenz aller Indivduen"'? (Zusatzfrage für
  philosophiehistorisch Gebildete: Wie verhält sich diese
  Unterscheidung zu der von Rousseau zwischen "`volonté
  générale"' und "`volonté de tous"' ?)
  
  \item \label{AufgPareto} Bei den Beweisen des "`Paradox des
  Liberalismus"'und des "`Satzes von Arrow"' wurde jeweils die schwache
  Pareto-Bedingung vorausgesetzt. Erkläre, warum sich die Beweise trotzdem
  genauso führen lassen, wenn man nur die {\em starke Paretobedinung}
  voraussetzt: Wenn kein Individuum eine bestimmte Alternative einer bestimmten
  anderen Alternative nachordnet, aber mindest ein Individuum sie vorzieht,
  dann sollte diese Alternative auch in der kollektiven Wahl bevorzugt werden.
  
  \item \label{AufgPL1} Zeige, dass die Gültigkeit des Beweises des "`Paradox
  des Liberalismus"' (Abschnitt \ref{LiberalismusParadox} auf Seite
  \pageref{LiberalismusParadox}ff.) nicht davon abhängt, über welche
  Alternativen man den beiden Individuen $A$ und $B$ ihre Prärogative einräumt.
  
  \item \label{AufgPL2} Zeige: Wenn umgekehrt zuerst die Präferenzen der
  Individuen festgelegt werden, und erst danach die Prärogative zugewiesen wird, 
  dann ist es immer
  möglich Prärogativen zu finden, so dass die Konstruktion einer
  kollektiven Entscheidungsfunktion für zwei Individuen und drei Alternativen
  doch möglich wird.

  \item \label{AufgArrow1} Gegeben seien die beiden Präferenzprofile $P_1$,
  $P_2$:
  \begin{center}
  \begin{tabular}{cccc}
                &  Individuum A         & Individuum B        & Individuum C \\
  Profil $P_1$  &  $y \succ x \succ a$  & $x \succ a \succ y$ & $y \succ a \succ x$ \\ 
  Profil $P_2$  &  $y \succ a \succ x$  & $a \succ x \succ y$ & $a \succ y \succ x$ \\
  \end{tabular}
  \end{center}
  und die Sozialwahlfunktionen $S_1$, $S_2$, $S_3$, $S_4$, $S_5$, $S_6$:
  \begin{center}
  \begin{tabular}{ccc}
               &  Profil $P_1$        & Profil $P_2$        \\
  Swf $S_1$    &  $y \succ x \succ a$ & $y \succ a \succ x$ \\  % ++
  Swf $S_2$    &  $y \succ a \succ x$ & $y \succ x \succ a$ \\  % ++
  Swf $S_3$    &  $x \succ y \succ a$ & $x \succ a \succ y$ \\  % +-
  Swf $S_4$    &  $a \succ y \succ x$ & $a \succ y \succ x$ \\  % ++
  Swf $S_5$    &  $x \succ a \succ y$ & $a \succ x \succ y$ \\  % ++
  Swf $S_6$    &  $a \succ x \succ y$ & $y \succ x \succ a$ \\  % --
  \end{tabular}
  \end{center}
  Aufaben:
  \begin{enumerate}
  \item Welche dieser Sozialwahlfunktionen erfüllt die Bedingung der
        {\em Unabhängigkeit von irrelevanten Alternativen}, welche nicht?
  \item Was ändert sich daran, wenn man Individuum C streicht?
  \end{enumerate}
 
  \item \label{AufgArrow2} Beweise:
  \begin{enumerate}
    \item Eine Sozialwahlfunktion, die jedem Präferenzprofil dieselbe
    "`soziale Wahl"' zuweist, ist immer mit der Bedingung der {\em
    Unabhängigkeit von irrelevanten Alternativen} vereinbar.
    \item Eine Sozialwahlfunktion, die jedem Präferenprofil die
    Präferenzordnung ein- und desselben Individuums aus dem Profil zuordnet,
    ist immer mit der Bedingung der {\em Unabhängigkeit von irrelevanten Alternativen} 
    vereinbar.
  \end{enumerate}
 
  \item \label{AufgArrow4} Zeige, wie man durch sukzessives Ersetzen des
  rechten und des linken Terms in der Aussage: "`$D$ ist entscheidend für $x$ über
  $y$"' die umgekehrte Aussage: "`$D$ ist entscheidend für $y$ über $x$"'
  ableiten kann. Wieviele Alternativen außer $x$ und $y$ benötigt man dafür
  mindestens bzw. höchstens? 
  
  \item {\em Obsolet!} Es gelte: a) Für ein gegebenes $x$ und
  jedes beliebiege $u$ sei $J$ entscheidend für $x \succ u$. Und b) Für ein gegebenes
  $y$ und ein beliebiges $u'$ sei $J$ weiterhin entscheidend für $u' \succ y$.
  
  Zeige, dass dann gilt: $J$ ist entscheidend für jedes beliebige Paar $v$, $w$
  und zwar sowohl $v \succ w$ als auch $w \succ v$.
  
  \item Warum kann man sich bei bei Lemma 1 des ersten
  Beweises des Satzes von Arrow nicht auf die Betrachtung des 1. Teils:
  ``Ersetzbarkeit von rechts'' (Seite \pageref{Lemma1ErsetzbarkeitVonRechts})
  beschränken und dann den Beweis analog zu Schritt 8. von Lemma 3 (Seite
  \pageref{Lemma3Schritt8}) abkürzen?
  
  \item Erkläre, warum ist das Resultat, dass das "`zentrale Individuum"' 
  Diktator über $A$ und $C$ ist, am Ende von Teil 2 des alternativen Beweises
  (Abschnitt \ref{AlternativerBeweis}, Seite \pageref{AlternativerBeweis}ff.) 
  unabhängig von der Reihenfolge, in der die
  Individuen beim "`Übergang"' (Teil 1 des Beweises) durchgezählt werden?
  
  \item Warum kann man am Ende von Teil 1 des zweiten Beweises (Abschnitt
  \ref{AlternativerBeweis}, Seite \pageref{AlternativerBeweis}ff.) nicht sagen,
  dass das "`zentrale Individuum"' entscheidend für $B$
  über $A$ (oder eine bliebige andere Alternative ist)? 
  
  Zusatzfrage: Könnte man am Ende von Teil 1 sagen, dass die Menge aller
  Individuen bis zum "`zentralen Individuum"', beinahe entscheidend für $B$
  über $A$ (oder anstelle von $A$ für irgend eine andere Alternative außer $B$)
  ist? 
  
  ~\\{\bf schwere Aufgaben:}
 
  \item \label{AufgArrow3} Führe den Beweis von Lemma 1 für {\em vollständig}
  entscheidende statt bloß beinahe entscheidende Mengen.
  
  \item \label{AufgArrow6} Warum lässt sich der Beweis von Lemma 2 nicht
  auf dieselbe Weise für vollständig
  entscheidende statt bloß für entscheidende Mengen führen?\footnote{Zumindest
  nicht ohne Weiteres, denn auf gewaltsame Weise lässt sich der Beweis immer
  noch führen, wenn man den Begriff der beinahe entscheidenden Menge implizit
  in Lemma 2 einführt und Lemma 3 mit in Lemma 2 aufnimmt\ldots} (Daraus ergibt
  sich, warum die Einführung des -- zunächst vielleicht ewtas kontraintuitiven
  -- Begriffes der {\em beinahe} entscheidenden Mengen sinnvoll ist.)
 
  \item \label{AufgArrow7} Warum ist bei Teil 2 des zweiten Beweises bei Punkt
  \ref{Teil2Punkt2} auf Seite \pageref{Teil2Punkt2} der 
  Hinweis "`Beschränkt man die Betrachtung auf alle Alternativen
  $\succeq_n B$"' notwendig? 
  
  (Zusatzfrage: Warum gilt die Erkenntnis,
  dass $A \succ B$, dann trotzdem ohne Einschränkung für alle Alternativen?)
 
  
  ~\\{\bf für Interessierte}:  

  \item Wieviele Individuuen und wieviele Alternativen muss es mindestens
  geben, damit der Satz von Arrow gilt?
  
  \item Angenommen, es gibt $n$ Individuen und $k$ Alternativen stehen zur
  Debatte.
  \begin{enumerate}
    \item Wieviele mögliche Präferenz{\em ordnungen} kann ein Individuum haben?
    \item Wieviele mögliche Präferenz{\em ordnungen} kann das Kollektiv haben?
    \item Wieviele mögliche Präferenz{\em profile} gibt es?
    \item Wieviele mögliche Sozialwahlfunktionen gibt es?
  \end{enumerate}
  
  \item Bei dem Beweis des Satzes von Arrow (Seite \pageref{BeweisArrow}ff.)
  sind wir immer von strikter Bevorzugung $\succ$ ausgegangen. Was ist in
  diesem Zusammenhang zur Möglichkeit der Indifferenz $~$ zwischen Alternativen
  zu sagen?

  \item Finde einen einfacheren Beweis für Lemma 3?  
   
  \item Kann man aus den drei alternativen Beweisen einen einzigen
  zusammenbauen, der kürzer und eleganter ist als alle drei?
  
\end{enumerate}

